\documentclass[11pt]{scrartcl}
\usepackage[none]{hyphenat}
\usepackage[sexy]{evan}
\usepackage{xcolor}
\usepackage{color}
\usepackage{tocbibind}
\usepackage{graphicx}
\usepackage{float}
\usepackage[T1]{fontenc}
\usepackage{lmodern}
\parindent = 0pt

\definecolor{dkgreen}{rgb}{0,0.6,0}
\definecolor{gray}{rgb}{0.5,0.5,0.5}
\definecolor{mauve}{rgb}{0.58,0,0.82}
\definecolor{ChadDarkBlue}{rgb}{.1,0,.2}  
\definecolor{ChadBlue}{rgb}{.1,.1,.5}  
\definecolor{ChadRoyal}{rgb}{.2,.2,.8}  
%\definecolor{ChadGreen}{rgb}{0,.35,.1}
%\definecolor{ChadGreen}{rgb}{0,.5,.25}  % Too bright
%\definecolor{ChadGreen}{rgb}{0,.4,.2}    % Still too bright
\definecolor{ChadGreen}{rgb}{0.5, 0, 0.2}    % Dark Green
%\definecolor{ChadRed}{rgb}{.8,.1,.2}    % Too bright
\definecolor{ChadRed}{rgb}{.5,0,.5}  % purple

\usepackage{hyperref}

\begin{document}
	\title{\color{ChadBlue}LS1101 Assignment}
	\author{\color{ChadRoyal}Priyanshu Mahato}
	\date{\today}
	\maketitle
	
	\section{Question 1}
	
	Write a short note on the difference between ‘Improvisation’ and ‘Adaptability’ based on ‘PICERAS’ as seven pillars of life as proposed by \emph{Daniel E Koshland Jr.}\\[2pt]
	
	\textbf{Answer:}\\[2pt]
	
	First, let us look at the definitions of \emph{Improvisation} and \emph{Adaptability}.
	\begin{enumerate}
		\item \underline{Improvisation}: Because a living system will inevitably be a small fraction of the larger
		universe in which it lives, it will not be
		able to control all the changes and vicissitudes of its environment, so it must have
		some way to change its program. It is this ability to change it's fundamental program, which is termed as \emph{Improvisation}.
		
		\item \underline{Adaptability}: Adaptability is a behavioral response that is part of the fundamental program of an organism. The behavioral manifestations of adaptability are a development of feedback and feedforward responses at the molecular level and are responses of living systems that allow survival in quickly changing environments.  
		
		\end{enumerate}
		
		Now, let us discuss the differences between Improvisation and Adaptability.
		
		Improvisation is a form of adaptability, but is too
		slow for many of the environmental hazards that a living organism must face. For example, a human that puts a hand into a
		fire has a painful experience that might be selected against in evolution but the individual needs to withdraw his hand from
		the fire immediately to live appropriately thereafter. That behavioral response to pain is essential to survival and is a fundamental response of living systems that we call feedback.
		Adaptability could arguably include improvisation, but improvisation is a mechanism to change the fundamental program,
		whereas adaptability is a behavioral response that is part of the program.
		
		In short, Improvisation is a form of adaptability but is too slow for many environmental hazards and it is distinguished from improvisation because the response is timely and does not involve a change of the program. 
		
		Using examples to  distinguish between the two, say a warm period changes to an ice age so that the program is less effective, the system will need to change its program in order to survive. In our current living systems, such changes can be achieved by the process of mutation plus selection (both natural and anthropogenic) that allows programs to be optimized for new environmental challenges that are to be faced (\emph{Improvisation}). On the other hand, say, an animal sees a predator, then it will respond to the danger with hormonal changes and escape behaviour (fight or flight: \emph{Adaptability}). \cite{Dan} \cite{TKS}
		
		\pagebreak
		
		\section{Question 2}
		
		According to ‘PICERAS’, ‘C’ stands for compartmentalization and ‘S’ stands for seclusion. Explain how compartmentalization helps seclusion within a cell.\\[2pt]
		
		\textbf{Answer:}\\[2pt]
		
		\textbf{Compartmentalization in cells: }Compartmentalization refers to the separation of spaces in the living system that allow for separate environments for necessary chemical processes. Compartmentalization is necessary to protect the concentration of the ingredients for a reaction from outside environment.
		
		Cells are not amorphous mixture of proteins, lipids and other molecules. Instead, all cells are comprised of well-defined compartments, each specializing in a particular function. In many cases subcellular processes may be described based on whether they occur at the plasma membrane, within the cytosol or within membrane bound organells such as the nucleus, golgi apparatus or even vesicular components of the membrane trafficking system like lysosomes and endosomes. Compartmentalization increases the efficiency of many subcellular processes by concentrating the required components to a confined space within the cell.
		
		\textbf{Seclusion: }It is the separation of chemical  pathways and the specificity of the effect of molecules, so that processes can function separately within the living system.in organism of earth proteins aid in seclusion because of their individualized structure that are specific for their function, so that they can efficiently act without affecting separate function.
		
		\textbf{For golgi apparatus and ER: }In cell near some organelles such as golgi body/golgi apparatus there is a particular zone where any other organelles such as ribosomes mitochondria etc. are absent and only smooth endoplasmic reticulum is present. This zone is called \emph{Zone of exclusion}. Since the golgi apparatus contains many hydrolytic enzymes to prevent this compartmentalization, it helps to secure a spot just for  golgi body so that the other organelles are not digested within the cell.
		
		\textbf{For nucleus : }The nucleus is a spatially organized compartment. The most obvious way in which this is achieved is at the level of chromosomes. The positioning of chromosomes with respect to nuclear landmarks and with respect to each other is both non-random and cell-type specific. This suggests that cells possess molecular mechanisms to influence the folding and disposition of chromosomes within the nucleus. The localization of many proteins is also heterogeneous within the nucleus.So this can be considered compartmentalization of nucleus keeping the chromosomes secluded from the other part of the cell.
		
		\emph{Compartmentalization creates appropriate microenvironments for these diverse processes, allows damage limitation, minimizes non-specific interactions and consequently increased cellular efficiency. This is how it helps in seclusion as it minimizes non-specific interactions.} \cite{Dan} \cite{TKS}
		
		\pagebreak
		
		\section{Question 3}
		
		“Instead of phospholipid bilayers, presence of ether-lipids in cell membranes helps a certain group of organisms to survive”. Write a short note on this statement.\\[2pt]
		
		\textbf{Answer:}\\[2pt]
		
		Ether lipids are a major structural component of cell membranes. The incorporation of ether-linked alkyl chains in phospholipids alters their physical properties and affects membrane dynamics. This is largely attributed to the lack of a carbonyl oxygen at the sn-1 position, which facilitates stronger intermolecular hydrogen bonding between the headgroups \cite{Lohner}.
		
		In addition to their structural roles, ether lipids can also act as biological signaling molecules. Oxidation of the vinyl-ether bond of plasmalogens by brominating species produced by an eosinophil peroxidase in activated eosinophils results in a $\alpha$-bromo fatty aldehyde that acts as a phagocyte chemoattractant for the recruitment of other leukocytes to sites of inflammation \cite{Thukkani} \cite{Albert}.
		
		The unique structural characteristics of the archaeal polar lipids, that is, the sn-glycerol-1- phosphate (G-1-P) backbone, \textbf{ether linkages}, and isoprenoid hydrocarbon chains, are in striking contrast to the bacterial characteristics of the sn-glycerol-3-phosphate (G-3-P) backbone, \textbf{ester linkages}, and fatty acid chains. This contrast in membrane lipid structures between archaea and bacteria has been termed the “lipid divide” \cite{Koga}. The chemical properties and physiological roles of archaeal lipids are often discussed in terms of the presence of the chemically stable ether bonds in thermophilic archaea \cite{Nakamo}. Because the ether bonds of archaeal lipids are for the most part not broken down under conditions in which ester linkages are completely methanolyzed, it is generally believed that the archaeal ether lipids are thermotolerant or heat resistant. This implies that thermophilic organisms are able to grow at high temperature due to the chemical stability of their membrane lipids.
		
		\begin{figure}[H]
			\centering
			\includegraphics[scale=0.5]{diagram.png}
			\caption{diphytanylglycerol (archaeol: archaeal diether lipid)}
			\label{figure:diagram1}
		\end{figure}
	
		Ether lipids (Figure \ref{figure:diagram1})are always present in the archaea that reside in high-temperature environments without exceptions, but the mesophilic archaea also have ether lipids. In fact, not only archaea but also certain thermophilic bacteria contain ether lipids. The thermophilic lipid candidates, in addition to the archaeal ether lipids, are the chemically stable monobranched fatty alcohol-containing diether lipids. These have been assumed to be thermophilic lipids because of their thermostability (unhydrolyzability) (diether or C–C bond in the long-chain diol or membrane-spanning nature (dicarboxylic acid) like tetraether lipids). 
		
		Because tetraether type, membrane-spanning polar lipids were first found in thermoacidophilic archaeon \cite{Rosa}, these lipids are considered thermophilic lipids. Tetraether lipids are extended as a result of their C40 hydrocarbon chains passing across the membrane bilayer. Thus, tetraether lipids link the leaflets of the lipid bilayer covalently and thus make the membrane rigid. This structure allows membranes to tolerate extreme conditions.  The hyperthermophilic \emph{Methanopyrus kandleri} (90°C \cite{Nakamo}) has also only diether-type polar lipids.
		
		But, on the other hand, \emph{Methanothermobacter thermautotrophicus} (65$^\circ$C) has both archaeol- and caldarchaeol-based lipids, while the mesophilic species of Methanobacterium (37$^\circ$C) has almost the same core lipid composition. Similarly, some archaea that have caldarchaeol-based (tetraether-type) polar lipids in addition to archaeol-based polar lipids grow above 85$^\circ$C, and there is one that grows at 20$^\circ$C. Some archaea have only archaeol-based (diether-type) polar lipids and grow below 40$^\circ$C, yet there is one that grows at 90$^\circ$C.
		
		\emph{Therefore, the distribution pattern of the archaeol- and caldarchaeol-based polar lipids make it clear that although these ether lipids facilitate the survival of thermophilic archaea in extreme temperature conditions, they are not absolutely required for tolerance of high temperature.}
		
		\pagebreak
		
		\section{Question 4}
		
		“Electrochemical gradient across a biological membrane determines the energy cost for an ion to travel through the membrane”. Write a short note on this statement.\\[2pt]
		
		\textbf{Answer:}\\[2pt]
		
		Electrochemical gradient: A gradient of electrochemical potential, usually for an ion that can move across a membrane. The gradient consists of two parts, the chemical gradient or difference in solute concentration across a membrane and the electrical difference in charge across a membrane. When there are unequal concentrations of an ion across a permeable membrane, the ion will move across the membrane from the area of higher concentration to the area of lower concentration through simple diffusion. Ions also carry an electric charge in the form of electric potential across the membrane. If there is an unequal distribution of charges across the membrane, then the difference in electric potential generates a force that drives ion diffusion until the charges are balanced on both sides of membrane .
		
		The generation of a trans-membrane electrical potential through ion movement across a cell membrane drives biological processes like nerve impulse conduction, muscle  contraction, hormone secretion and sensory processes. By convention, a typical animal cell has a trans-membrane electrical potential of -50 $mV$ to -70 $mV$ inside the cell relative to the outside.
		
		Electrochemical gradient also plays a role in establishing proton gradients in oxidative phosphorylation inside mitochondria , the final step of respiration ETS(electron transport system)four complexes embedded in the inner membrane of mitochondria make up that electron transport chain.
		
		Again,the ions are also charged they cannot pass through the membrane via simple diffusion . Two diffusion . Two different mechanisms can transport the ions is the $Na^{+}$, $K^{+}$, Adenosine tri-phosphatase(ATP\emph{ase}). ATP\emph{ase} catalyzes the ATP to ADP and Inorganic Phosphate releasing an amount of energy,for every molecule of ATP hydrolysed 3 Sodium $Na^{+}$ ions are transported out of the membrane and 2 Potassium ions ($K^{+}$) get transported in the cell membrane. This creates a temporary positive and negative potential difference outside and inside of the membrane/cell respectively and potential difference is measured to be around -50 $V$ to -60 $V$.
		
		There are some other metabolic reactions where the Electrochemical Gradient takes place:
		
		\begin{enumerate}
			\item \underline{Bacteriohodopsin}: generates a proton gradient using proton pumps in archaebacteria 
			
			\item \underline{Photophosphorylation}: In the light phase of the photosynthesis PsII acts are electron carriers with other electron carriers like plastoquinone, plastocyanin, cytochrome B-6 f complex etc. to create electrochemical gradient which is used in the production of oxygen.
		\end{enumerate}
	
	\pagebreak
		
		
		
		\begin{thebibliography}{}
			\bibitem{Lohner} Lohner K. ``Is the high propensity of ethanolamine plasmalogens to form non-lamellar lipid structures manifested in the properties of biomembranes?'' \emph{Chem Phys Lipids}. 1996;81:167–184. doi: 10.1016/0009-3084(96)02580-7. [\href{https://www.ncbi.nlm.nih.gov/pubmed/8810047}{PubMed}] [\href{https://dx.doi.org/10.1016%2F0009-3084(96)02580-7}{CrossRef}] [\href{https://scholar.google.com/scholar_lookup?journal=Chem+Phys+Lipids&title=Is+the+high+propensity+of+ethanolamine+plasmalogens+to+form+non-lamellar+lipid+structures+manifested+in+the+properties+of+biomembranes?&author=K+Lohner&volume=81&publication_year=1996&pages=167-184&pmid=8810047&doi=10.1016/0009-3084(96)02580-7&}{Google Scholar}]
			
			\bibitem{Thukkani} Thukkani AK, Hsu F-F, Crowley JR, et al. ``Reactive chlorinating species produced during neutrophil activation target tissue plasmalogens: Production Of The Chemoattractant, 2-Chlorohexadecanal.'' \emph{J Biol Chem}. 2002;277:3842–3849. doi: 10.1074/jbc.M109489200. [\href{https://www.ncbi.nlm.nih.gov/pubmed/11724792}{PubMed}] [\href{https://dx.doi.org/10.1074%2Fjbc.M109489200}{CrossRef}] [\href{https://www.ncbi.nlm.nih.gov/pmc/articles/PMC5818364/#CR88}{RefList}]
			
			\bibitem{Albert} Albert CJ, Thukkan AK, Heuertz RM, et al. ``Eosinophil peroxidase-derived reactive brominating species target the vinyl ether bond of plasmalogens generating a novel chemoattractant, alpha -bromo fatty aldehyde.'' \emph{J Biol Chem}. 2003;278:8942–8950. doi: 10.1074/jbc.M211634200. [\href{https://www.ncbi.nlm.nih.gov/pubmed/12643282}{PubMed}] [\href{https://dx.doi.org/10.1074%2Fjbc.M211634200}{CrossRef}] [\href{https://scholar.google.com/scholar_lookup?journal=J+Biol+Chem&title=Eosinophil+peroxidase-derived+reactive+brominating+species+target+the+vinyl+ether+bond+of+plasmalogens+generating+a+novel+chemoattractant,+alpha+-bromo+fatty+aldehyde&author=CJ+Albert&author=AK+Thukkan&author=RM+Heuertz&volume=278&publication_year=2003&pages=8942-8950&pmid=12643282&doi=10.1074/jbc.M211634200&}{Google Scholar}]
			
			\bibitem{Yosuke} Yosuke Koga, ``Thermal Adaptation of the Archaeal and Bacterial Lipid Membranes'', \emph{Archaea}, vol. 2012, Article ID 789652, 6 pages, 2012. \url{https://doi.org/10.1155/2012/789652} [\href{https://www.researchgate.net/publication/230749321_Thermal_Adaptation_of_the_Archaeal_and_Bacterial_Lipid_Membranes?_sg%5B1%5D=zN9asPMgXeArrcCwtMWMJQ2CBvdV4aYmYzHm1bEwOTsZsfLwXkaelxrj_lfYaOmcMk5Ca1islg}{ResearchGate}]
			
			\bibitem{Koga} Y. Koga, “Early evolution of membrane lipids: how did the lipid divide occur?” \emph{Journal of Molecular Evolution}, vol. 72, no. 3, pp. 274–282, 2011. [\href{https://link.springer.com/article/10.1007/s00239-011-9428-5}{Springer}] [\href{https://scholar.google.com/scholar_lookup?title=Early%20evolution%20of%20membrane%20lipids:%20how%20did%20the%20lipid%20divide%20occur?&author=Y.%20Koga&publication_year=2011}{Google Scholar}]
			
			\bibitem{Nakamo} Y. Koga and M. Nakano, “A dendrogram of archaea based on lipid component parts composition and its relationship to rRNA phylogeny,” \emph{Systematic and Applied Microbiology}, vol. 31, no. 3, pp. 169–182, 2008. [\href{https://www.sciencedirect.com/science/article/abs/pii/S0723202008000246?via%3Dihub}{ScienceDirect}] [\href{https://scholar.google.com/scholar_lookup?title=A%20dendrogram%20of%20archaea%20based%20on%20lipid%20component%20parts%20composition%20and%20its%20relationship%20to%20rRNA%20phylogeny&author=Y.%20Koga%20&author=M.%20Nakano&publication_year=2008}{Google Scholar}]
			
			\bibitem{Rosa} M. de Rosa, S. de Rosa, A. Gambacorta, L. Minale, and J. D. Bu'lock, “Chemical structure of the ether lipids of thermophilic acidophilic bacteria of the \emph{Caldariella} group,” \emph{Phytochemistry}, vol. 16, no. 12, pp. 1961–1965, 1977. [\href{https://scholar.google.com/scholar_lookup?title=Chemical%20structure%20of%20the%20ether%20lipids%20of%20thermophilic%20acidophilic%20bacteria%20of%20the%20Caldariella%20group&author=M.%20de%20Rosa&author=S.%20de%20Rosa&author=A.%20Gambacorta&author=L.%20Minale&author=&author=J.%20D.%20Bu%27lock&publication_year=1977}{Google Scholar}]
			
			\bibitem{Dan} Daniel E. Koshland, Jr., ``The Seven Pillars of Life'' \emph{Science}, vol 295, issue 5563, pp. 2215-2216, 22 Mar 2002. \url{https://www.science.org/doi/10.1126/science.1068489} [\href{https://www.pnas.org/doi/full/10.1073/pnas.0707644104}{PNAS}] [\href{http://nasonline.org/member-directory/deceased-members/53659.html}{NAS}] 
			
			\bibitem{citekey} \href{https://www.wikipedia.org/}{Wikipedia}
			
			\bibitem{TKS} Notes and lecture slides provided by our LS1101 instructor, \emph{\href{mailto:senguptk@iiserkol.ac.in}{Prof. Tapas Kumar Sengupta}}
		\end{thebibliography}		

\end{document}