\documentclass[12pt]{article}

\usepackage[margin=1in]{geometry}
\usepackage{amsfonts, amssymb, amsmath}
\usepackage[none]{hyphenat}
\usepackage{fancyhdr}
\usepackage{float}
\usepackage{setspace}
\usepackage{graphicx}
\usepackage{hyperref}
\usepackage{xcolor}

\pagestyle{fancy}
\fancyhead{}
\fancyfoot{}
\fancyhead[L]{\MakeUppercase{CH1101 PS-3}}
\fancyhead[R]{\slshape Priyanshu Mahato: \href{mailto:pm21ms002@iiserkol.ac.in}{\color{purple}pm21ms002@iiserkol.ac.in}}
\fancyfoot[C]{\thepage}

\renewcommand{\footrulewidth}{1pt}
% \renewcommand{\baselinestretch}{1.2}

\setlength{\headheight}{16pt}
\setlength{\parindent}{0em}
\setlength{\parskip}{3em}

\begin{document}
	\thispagestyle{empty}
	
	\begin{titlepage}
		\begin{center}
			\vspace{2cm}
			\huge\textbf{CH1101}\\
			\vspace{1cm}
			\large\textbf{Elements of Chemistry}
			\vfill
			\line(1, 0){470}\\[14pt]
			\huge\textbf{Problem Sheet - 3}\\[10pt]
			\Large\textbf{Structure of the Atom}\\[14pt]
			\line(1, 0){470}
			\vfill
			By: Priyanshu Mahato\\
			Roll No.: pm21ms002\\
			\today
		\end{center}
	\end{titlepage}

	\setcounter{page}{1}
	
	\textbf{Question 1: }Sketch the 2s, 3s and 4s orbitals of H-atom. Label the axis, nucleus, radial node, nodal plane and sign of wavefunction, wherever applicable.\\
	
	\textbf{Answer: } For 2s,\\
	
	\begin{figure}[H]
		\centering
		\includegraphics[scale=0.5]{2s}
		\caption{2s}
	\end{figure}

	For 3s,\\

	\begin{figure}[H]
		\centering
		\includegraphics[scale=0.5]{3s}
		\caption{3s}
	\end{figure}

	For 4s,\\
	
	\begin{figure}[H]
		\centering
		\includegraphics[scale=0.5]{4s}
		\caption{4s}
	\end{figure}

	\textbf{Question 2: }Sketch the $2p_{x}$, $3p_{z}$ and $4p_{y}$ (draw only one equal probability iso-surface; NOT contour plots). Label the axis, nucleus, radial node, nodal plane and sign of wavefunction, wherever applicable.\\
	
	\textbf{Answer: }

	\begin{figure}[H]
		\centering
		\includegraphics[scale=0.5]{2px}
		\caption{$2p_{x}$}
	\end{figure}

	\begin{figure}[H]
		\centering
		\includegraphics[scale=0.5]{3pz}
		\caption{$3p_{z}$}
	\end{figure}

	\begin{figure}[H]
		\centering
		\includegraphics[scale=0.5]{4py}
		\caption{$4p_{y}$}
	\end{figure}

\pagebreak

	\textbf{Question 3: }Draw up a table showing the number of radial nodes, the number of angular nodes (nodal planes), and the total number of radial and angular nodes for 1s, 2s, 2p, 3s and 3p orbitals.
	
	\textbf{Answer: }
	
	\begin{table}[H]
		\centering
		\def\arraystretch{1.5}
		\begin{tabular}{|c|c|c|c|}
			\hline
			$Orbital$&No. of Radial Nodes&No. of Angular Nodes&Total no. of Nodes\\
			\hline
			$1s$&$0$&$0$&$0$\\
			\hline
			$2s$&$1$&$0$&$1$\\
			\hline
			$2p$&$0$&$1$&$1$\\
			\hline
			$3s$&$2$&$0$&$2$\\
			\hline
			$3p$&$1$&$1$&$2$\\
			\hline
		\end{tabular}
	\end{table}

	\textbf{Question 4: }For the Li atom, compare the 2s and 2p electrons in terms of shielding ability? Explain.
	
	\textbf{Answer: }First of all, let us look at the RPD of $n=2$ electrons in Li$^{2+}$ atom,\\
	
	\begin{figure}[H]
		\centering
		\includegraphics[scale=0.7]{RPD}
		\caption{RPD of $n=2$ electrons of the $Li^{2+}$ electrons}
	\end{figure}

	From the graph, it is clear that most of the probability of finding 2s and 2p electrons lies outside the shielding area of 1s electrons. But, 2s has a local maximum closer to the nucleus, most of which, lies inside the region occupied by 1s. This tells us that 1s doesn't shield 2s electrons completely. However, a very insignificant portion of 2p (the 'tail' of the RPD) lies in the 1s region. Thus, 1s shields 2p electrons far better than it does in the case of 2s electrons. We can thus say that 2s experiences greater effective nuclear charge due to the possibility of being inside the 1s screen, and 2p experiences much less effective nuclear charge. We can also deduce that the overall energy of 2s is lower than 2p (but $|E_{2s}| > |E_{2p}|$). Therefore, 2s can shield electrons (from nuclear charge) better than 2p.
	
	\textbf{Question 5: }The contour plot shown below is of one of the 4p orbitals: positive intensity is indicated 
	by red contours, negative by blue, and the zero contour is indicated in green.
	Sketch how the wavefunction will vary along the dotted line a, and along the two 
	circular paths b and c (for the latter two, this means making a plot of the wavefunction 
	as a function of an angle which specifies how far we have moved around the circle).
	
	\begin{figure}[H]
		\centering
		\includegraphics[scale=0.7]{Q5}
	\end{figure}

	\textbf{Answer: }P.T.O.\\
	
	\begin{figure}[H]
		\centering
		\includegraphics[scale=0.16]{Along_a}
		\caption{Along $a$}
	\end{figure}

	\begin{figure}[H]
		\centering
		\includegraphics[scale=0.12]{Along_b}
		\caption{Along $b$}
	\end{figure}

	\begin{figure}[H]
		\centering
		\includegraphics[scale=0.1]{Along_c}
		\caption{Along $c$}
	\end{figure}

	\textbf{Question 6: } The contour plot of wavefunction ($\Psi$) shown below is of one of the p orbitals. Consider that 
	the contour plot of $\Psi^{2}$ looks like the plot shown below (of course, there will be no “signs”). In 
	such a scenario, which of the following statements is true:\\
	
	\begin{figure}[H]
		\centering
		\includegraphics[scale=0.5]{Q6}
	\end{figure}

	(A) Between “c”, “d” and “e”, the ``probability of finding electron" increases in the order 
	$c>d>e$\\
	(B) The circle “b” represents the angular node\\
	(C) The line “a” represents the radial node\\
	(D) The “probability of finding electrons” is higher in the point “d” in comparison to “f”\\
	
	\textbf{Answer: }(D) is  the correct answer.
\end{document}