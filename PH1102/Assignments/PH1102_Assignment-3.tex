\documentclass{article}

\usepackage[margin=1in]{geometry}
\usepackage{amsfonts, amsmath, amssymb}
\usepackage[none]{hyphenat}% prevents LaTeX from adding hyphenated words
\usepackage{fancyhdr} %helps introduce fancy headers and footers in the document
\usepackage{graphicx}
\usepackage{float}
\usepackage{setspace}
\usepackage[nottoc, notlot, notlof]{tocbibind}
\usepackage{hyperref}
\usepackage{xcolor}
\usepackage{color}

\definecolor{dkgreen}{rgb}{0,0.6,0}
\definecolor{gray}{rgb}{0.5,0.5,0.5}
\definecolor{mauve}{rgb}{0.58,0,0.82}

\pagestyle{fancy}
\fancyhead[L]{PH1102 Experiment - 3}
\fancyhead[R]{Priyanshu Mahato: \color{mauve} \href{mailto:pm21ms002@iiserkol.ac.in}{pm21ms002@iiserkol.ac.in}}
\fancyfoot[C]{\thepage}
\setlength\parindent{0em}

\begin{document}
	\thispagestyle{empty}
	\begin{titlepage}
		\begin{center}
			\vspace{1cm}
			\Large\textbf{Physics Laboratory PH1102}\\
			\vspace{1cm}
			\large\textbf{Lab Report}
			\vfill%fills rest of the page with spaces and adjusts as you add elements to the page
			\line(1,0){450}\\[16pt]%creates a line
			\huge\textbf{Experiment No.: 03}\\[10pt]
			\Large\textbf{- STOKES LAW AND VISCOSITY -}\\[15pt]
			\line(1,0){450}
			\vfill
			\begin{figure}[H]
				\centering
				\includegraphics[scale=0.15]{IISER-K_Logo.png}
			\end{figure}
			By : Priyanshu Mahato\\
			Roll No. : pm21ms002\\
			\today\\
		\end{center}
	\end{titlepage}

	\setcounter{page}{0}
	\tableofcontents
	\pagebreak

	\section{Aim}
	
	To determine the coefficient of viscosity of a liquid using a falling ball viscometer 
	(employing Stokes’ law).
	
	\section{Experimental apparatus}
	
	The main experimental apparatus consists of a glass tube containing the experimental 
	liquid. \textbf{P} and \textbf{Q} are two adjustable reference marks along the tube length. The entire glass 
	system is supported by a wooden stand. Besides this you will need a few steel balls of 
	different diameter, screw gauge, digital balance, vernier calliper, stop watch, and meter scale 
	for this experiment.
	
	\section{Principle of the Experiment}
	
	A body moving in a fluid is acted on by a frictional force in the opposite direction of its 
	velocity. The magnitude of this force depends on the geometry of the body, its velocity, and 
	the internal friction of the fluid. A measure for the internal friction is given by the dynamic 
	viscosity \textbf{$\eta$}. For a spherical ball of radius \textbf{$r$} moving at velocity \textbf{$v$} in an infinitely extended 
	fluid of dynamic viscosity \textbf{$\eta$}, G. G. Stokes derived the viscous force to be:
	\begin{align}
		F_{v} = 6\pi \eta r v
	\end{align}
	If the spherical ball (density $\rho$) is dropped from rest at the upper surface of a vertical liquid (density $\sigma$ ) column , gravitational force $F_{g} = \frac{4}{3} \pi r^{3} \rho g$ ($g=$ acceleration due to gravity) and buoyancy force $F_{b} = \frac{4}{3} \pi r^{3} \sigma g$ act on it besides viscous force $F_{v} = 6 \pi \eta r v$.
	
	Initially, $F_{v}$ is zero because ball started at rest. So, the ball accelerates downwards because 
	there is a net downward force. Then $F_{v}$ starts to increase. Eventually a force balance 
	$F_{v} + F_{b} = F_{g}$ is reached and the ball attains a steady terminal velocity $v_{t}$. Using the force 
	balance condition one can easily derive,
	\begin{align}
		\eta = \frac{2}{9} \frac{r(\rho - \sigma)g}{v_{t}}
	\end{align}
	Actually, Eq. (1) [derived under the assumption of infinitely extended liquid] should be 
	corrected for the finite size of the liquid column. For the movement of the spherical ball 
	along the axis of a liquid cylinder of radius $R$ and length $h$, the viscous force is,
	\begin{align*}
		F_{v} = 6\pi \eta rv \left(1 + 3.3\frac{r}{h}\right) \left(1 + 2.4 \frac{r}{R}\right)
	\end{align*}
	For the experimental situation in our lab, $\frac{r}{R} \approx 0.1$ and $\frac{r}{h} \approx 0.001$. Thus, finite length correction $\left(1 + 3.3\frac{r}{h}\right)$ may be ignored. Incorporating the correction due to finite radius of the 
	liquid column, Eq. (2) becomes:
	\begin{align}
		\eta = \frac{2}{9} \frac{r^{2} (\rho - \sigma)g}{v_{t} \left(1 + 2.4 \frac{r}{R}\right)}
		\label{equation:tits}
	\end{align}
	\section{Diagram}
	
	\begin{figure}[H]
		\centering
		\includegraphics[scale=0.7]{diagram.png}
		\caption{Schematic Diagram}
	\end{figure}

	\pagebreak
	
	\section{Results and Observations}
	
	Tables:
	
	\begin{enumerate}
		\item Determination of radius and mass of the spherical balls (for three different radii)\\
		Density of Castor oil = 0.961 $g \cdot cc^{-1}$\\
		Today's temperature = $23 ^{\circ} C$\\
		Least count of the Screw Gauge = 0.001 cm\\
		Least count of Digital Balance = 0.01 g\\
		Least count of stopwatch = 0.01 s\\
		
		\begin{enumerate}
			\item Ball 1 :
			\begin{figure}[H]
				\centering
				\includegraphics[scale=0.5]{table 1.png}
			\end{figure}
		
			\item Ball 2 :
			\begin{figure}[H]
				\centering
				\includegraphics[scale=0.5]{table 2.png}
			\end{figure}
		
			\pagebreak
		
			\item Ball 3 :
			\begin{figure}[H]
				\centering
				\includegraphics[scale=0.5]{table 3.png}
			\end{figure}
		\end{enumerate}
	
	\item Measurement of terminal velocity\\
	Distance travelled by Balls is 80 cm = 0.8 m\\
	
	\begin{figure}[H]
		\centering
		\includegraphics[scale=0.5]{table 4.png}
	\end{figure}

	\item Determination of the Inner Diameter of Glass Cylinder\\
	Vernier Constant of Vernier Calliper = 0.002 cm\\
	
	\begin{figure}[H]
		\centering
		\includegraphics[scale=0.5]{table 5.png}
	\end{figure}
	\end{enumerate}

	\pagebreak

	\section{Plot}
	
	Plot of Velocity ($v$) in $ms^{-1}$ vs. Square of Radius ($r^{2}$) in $cm^{2}$:
	\begin{figure}[H]
		\centering
		\includegraphics[scale=0.5]{plot 1.png}
	\end{figure}

	\section{Inference}
	
	From the above graph, we see that with the given set of data, the resultant line is not a 
	straight line. To remove it, we need to find the best possible fit of the line. If we fit the line 
	within the given data-points in the best possible manner, the slope of the line comes out to be 243.615 $cm^{-1}s^{-1}$.\\
	The fitted graph is:
	
	\begin{figure}[H]
		\centering
		\includegraphics[scale=0.5]{plot 2.png}
	\end{figure}

	This value is the value of the expression: $\frac{2}{9} \frac{(\rho - \sigma)g}{\eta \left(1 + 2.4 \frac{r}{R}\right)}$\\
	
	According to the tables, the average value of r is : 0.1812 cm. The value of R is 2.305 cm. 
	
	Hence, substituting the values (taking the density of Castor oil as 0.961 $g \cdot cc^{-1}$ and density of 
	the material as the average value 10.29 $g \cdot cc^{-1}$), we get the value of $\eta$ as : 6.35 $poise$ $=$ 0.635 $Pa \cdot s$
	
	The given value of co-efficient of viscosity of Castor oil is 0.650 $Pa \cdot s$
	
	\section{Error Analysis}
	
	\textbf{The error in the calculation of $\eta$ is :} 
	\begin{align*}
		Absolute\ Error &= |0.650 - 0.635|\\
		&= 0.015\ Pa \cdot s\\
		Percentage\ Error &= \frac{0.015}{0.650} \times 100 \%\\
		&= 2.3 \%
	\end{align*}

	\textbf{Systematic/Instrumentation Error}
	
	We used Equation \ref{equation:tits} to find the value of $\eta$. In the process, errors of the measured quantities get accumulated and propagated to give a \emph{systematic error} in the value of $\eta$. Using Partial Derivatives in the Equation \ref{equation:tits}, we get,
	$$\frac{\delta\eta}{\eta} = \left|\frac{2}{r} - \frac{2.4}{R + 2.4r} - \frac{9m}{4\pi r^{4} (\rho - \sigma)}\right| \delta r + \left|\frac{1}{x}\right| \delta x + \left|\frac{1}{t}\right| \delta t + \left|\frac{2.4r}{R(R + 2.4r)}\right| \delta R + \left|\frac{3}{4 \pi r^{3}(\rho - \sigma)}\right| \delta m$$
	
	Now, substituting different parameters and least counts of the instruments, as individual errors in the above formula, we obtain,
	
	\begin{enumerate}
		\item \textbf{Ball 1:} \\
		For Ball 1, the systematic error comes out to be about 9.8\%
		\item \textbf{Ball 2:}\\
		For Ball 2 the systematic error comes out to be about 6.6\%
		\item \textbf{Ball 3:}\\
		For Ball 3 the systematic error comes out to be about 3.4\%
	\end{enumerate}
	
	\section{Sources of Error}
	
	\begin{enumerate}
		\item The point where the glass bead attains the terminal velocity is an eye-estimated length and 
		has 100$\%$ chance for errors.
		
		\item The inaccuracy in screw-gauge, vernier-calliper and digital balance can contribute to the errors of 
		the constituting variables.
		
		\item Inaccuracy in working with the stopwatch as the spherical watch passes the two reference marks (namely, \textbf{P} and \textbf{Q}).
	\end{enumerate}
		\section{Conclusion}
		
		The Experimental value of the co-efficient of viscosity of Castor Oil differs from the actual 
		one by 2.3 $\%$.
\end{document}