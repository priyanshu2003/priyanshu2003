\documentclass[11pt]{scrartcl}
\usepackage{answers}
\Newassociation{sketch}{hintitem}{hints}
\renewcommand{\solutionextension}{out}
\usepackage[none]{hyphenat}
\usepackage[margin=1in]{geometry}

\usepackage[sexy]{evan}
\newcommand\EE{\mathbb E}
\newcommand\PP{\mathbb P}
%\setlength{\parindent}{0em}
%\setlength{\parskip}{2.5em}

\begin{document}
	\title{Module 2: Number Systems}
	\author{Priyanshu Mahato}
	\date{\today}
	\maketitle
	
	\begin{abstract}
		Email: \mailto{pm21ms002@iiserkol.ac.in}.These are my personal notes on Number Systems. We will consider $\mathbb{N}$ (Natural Numbers), $\mathbb{Z}$ (Integers), and $\mathbb{Q}$ (rational Numbers), but not $\mathbb{R}$ (Real Numbers).
	\end{abstract}

	\section{Natural Numbers}
	
		The natural numbers are $1,2,3,4,\dots$. The set of all natural numbers is denoted by $\NN$.
		
		\begin{definition}
			We assume familiarity with the algebraic operations of addition and multiplication on the set $\NN$ and also with the linear order relation $<$ on $\NN$ defined by ``$a<b$ if $a, b \in \NN$ and $a$ is less than $b$".
		\end{definition}
	
		We discuss the following fundamental properties of the set $\NN$.
		
		\begin{enumerate}
			\item Well Ordering Property
			\item  Principle of Induction
		\end{enumerate}
	
	\subsection{Well Ordering Property}
		\begin{definition}
			Every non-empty subset of $\NN$ has a least element.
		\end{definition}
		This means that if $S$ is a non-empty subset of $\NN$, then there is an element $m$ in $S$ such that $m \leq s$ for all $s \in S$.
		
		In particular, $\NN$ itself has the least element 1.
		
		\begin{proof}
			Let $S$ be a non-empty subset of $\NN$. Let $k$ be an element of $S$. Then $k$ is a natural number.
			
			We define a subset $T$ by $T = \left\{ x \in S : x \leq k \right\}$. The $T$ is a non-empty subset of $\left\{ 1,2,3,\dots, k \right\}$.
			
			Clearly, $T$ is a finite subset of $\NN$ and therefore it has a least element, say $m$. Then $1 \leq m \leq k$.
			
			We now show that $m$ is the least element of $S$. Let $s$ be any element of $S$.
			
			If $s>k$, then the inequality $m \leq k$ implies $m<s$.
			
			If $s \leq k$, the $s \in T$; and $m$ being the least element of $T$, we have $m \leq s$.
			
			Thus $m$ is the least element of $S$.
		\end{proof}
	
	\subsection{Principle of Induction}
	\begin{definition}
		Let $S$ be a subset of $\NN$ such that,
		\begin{enumerate}[i)]
			\item  $1 \in S$ and,
			\item  if $k \in S$, then $k+1 \in S$.
		\end{enumerate}
	Then $S = \NN$
	\end{definition}
	
	\begin{proof}
		Let $T = \NN - S$. We prove that $T = \phi$.
		
		Let $T$ be non-empty. then by the \emph{Well Ordering Property} of $\NN$, the non-empty subset $T$ has a least element, say $m$.
		
		Since $1 \in S$ and $1$ is the least element of $\NN$, $m>1$.
		
		Hence, $m-1$ is a natural number and $m-1 \notin T$. So, $m-1 \in S$.
		
		But by ii) $m-1 \in S \Rightarrow (m-1)+1 \in S$, i.e., $m \in S$.
		
		This contradicts that $m$ is the least element in $T$. Therefore, our assumption is wrong and $T = \phi$.
		
		Therefore, $S=\NN$.
	\end{proof}
		 
	\begin{theorem}
		Let $P(n)$ be a statement involving a natural number $n$. If,
		\begin{enumerate}[i)]
			\item $P(1)$ is true, and
			\item $P(k+1)$ is true whenever $P(k)$ is true,
		\end{enumerate}
		then $P(n)$ is true for all $n \in \NN$.
	\end{theorem}

	\begin{proof}
		
	\end{proof}
\end{document}

